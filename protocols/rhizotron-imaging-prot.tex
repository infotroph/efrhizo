\documentclass[11pt]{article}
\usepackage{graphicx}
\usepackage{multirow}
\usepackage[tiny,compact]{titlesec}

\title{Minirhizotron Image Collection at the Energy Farm}
\author{Chris Black, black11@igb.uiuc.edu\\503 929-9421}
\date{Rev. 2, 29 March 2010}

\begin{document}
\maketitle


\section{Equipment}
	\begin{itemize}
		\item{Rhizotron camera, with case}
		\item{I-Cap laptop}
		\item{batteries (at least 2 for a full day)}	
		\item{Swab stick with short-pile and long-pile rollers}
		\item{Water Sensor}
		\item{Paper towels}
		\item{To clean the camera if if gets wet: Q-tips, paper towels, alcohol, masking tape}
		\item{Flashlight}
		\item{Plot map with tube numbers}
	\end{itemize}
	
\section{[mis-] Labeling} 
	There are 96 tubes, with one beside each root collar in the small plots (blocks 1-4) and one beside 8 out of 12 collars (the ones without root exclosures) in the large plots (block 0). Originally, the tubes were labeled by block, crop, and collar (e.g. `3S2'), but iCap thinks in tube numbers (e.g. `66'), so we're re-labeling with tube number as we replace tubes. Tubes that have been replaced since 2009 are now labeled with their tube number, while original tubes are still labeled with the block/crop/collar code. Tube numbers are ordered by block and collar within crops: corn is tubes 1-24, \textit{Miscanthus} 25-48, switchgrass 49-72, and prairie 73-96. The camera case is `tube 97', to be used for calibration images. 
	
	There are at least two tubes that are labeled incorrectly: 0S11 is labeled in black as 0S12 (True 0S12 is labeled in blue), and both 4c1 and 4c2 are labeled as 4c1.
	
	Note that the labeling for Corn block 0 does not follow the usual counterclockwise collar numbering: 0C1 through 0C4 form an east-to-west line in the north half of the field, and 0C5 through 0C8 form a (ragged) east-to-west line in the south half of the field.
	
	My plot maps show all of this in simpler terms. Sorry for the confusion.

%%% Not very useful. 
%%% The plot maps are really necessary for anything more detailed 
%%% than crop level.
%\begin{table}[htbp]
%	\begin{center}
%		\caption{\label{conversion}Tube label to tube number conversion.}
%		\vskip1em
%		\begin{tabular}{rll|rll}
%Crop & Block & Tubes & Crop & Block & Tubes\\
%\hline
%Corn		& 0 & 1--8		& Switchgrass & 0 & 49--56\\
%	 		& 1 & 9--12		&			  & 1 & 57--60\\
%	 		& 2 & 13--16	&			  & 2 & 61--64\\
%	 		& 3 & 17--20	&			  & 3 & 65--68\\
%	 		& 4 & 21--24	&			  & 4 & 69--72\\
%Miscanthus & 0 & 25--32	& Prairie	  & 0 & 73--80\\
%	 		& 1 & 33--36	&			  & 1 & 81--84\\
%	 		& 2 & 37--40	&			  & 2 & 85--88\\
%	 		& 3 & 41--44	&			  & 3 & 89--92\\
%	 		& 4 & 45--48	&			  & 4 & 93--96\\
% 			&&&\multicolumn{2}{r}{Case (for calibration)}& 97\\
%		\end{tabular}
%	\end{center}
%\end{table}

\section{Experiment setup}
	Open I-Cap and press ``Open an existing database." Select `EF(EHD).btc' and press ``Open''. If it opens without a hitch, enter your initials and check that the gather direction is DOWN, then move on to image capture.
	If I-Cap crashes with a `subscript out of range' message, don't panic. This is from  a bug in the image transfer software, and not your fault. If you're already in the field  when you discover this, just set up a new experiment and we'll merge the images back into EF later.
\begin{enumerate}
	\item{Select ``Create a new experiment database.''}
	\item{Give the experiment a name something like "EF2009AUG29".}
	\item{Principal Investigator: EHD. Total Number of Tubes: 97. Image Spacing: 13.5. Tube Length: 1620 mm.}
	\item{Image Compression: JPG. Image Captioning: Off.}
	\item{Ignore the proposed start and end dates.}
	\item{Press ``Next''}
	\item{Leave the Quick Notes fields blank.}
	\item{Enter your initials and set the gather direction to DOWN.}
	\item{Save a calibration image.}
	\item{Proceed with image capture.}
\end{enumerate}

		
\section{Calibration}
	The camera's widest zoom setting leaves excessive vignetting, so it's necessary to zoom in slightly, but the zoom button doesn't give a reproducibly exact zoom level. Thus, calibrate. 
	
\begin{enumerate}
	\item{Tell I-Cap to go to tube 97. Go to the location with the same number as the next tube you'll image (e.g. 73 if you're about to image 0P2).}
	\item{Use the ``Notes'' feature, \textit{before} capturing an image, to record which tubes this calibration is good for. Use the form ``Calibration image for tubes x-y''.}
	\item{Put the camera into its carrying case and rotate it until you can see the 14x18 mm grid of lines spaced 1 mm apart.}
	\item{Zoom all the way in on any part of the grid, focus until the ink spatters around the lines show crisply, then zoom all the way back out.}
	\item{Adjust the zoom until the image area is as close as possible to 12x17 mm. (Realistically, that means somewhere between 16 and 18 mm horizontal.}
	\item{Save the image.}
	\item{Move the camera to the correct tube number and take images. Do not disturb the zoom setting. Adjust the focus setting as needed, but it should be okay as it is.}
	\item{Always work in order of tube number, so that if the zoom level gets disturbed you can easily add a new calibration image.} 
\end{enumerate}
	
	One calibration per bout of image collection should be enough, \textit{if} you're certain that you haven't touched the zoom button since calibrating. More calibrations, if they're clearly labeled and saved at locations corresponding to the tubes for which they were used, won't hurt anything (other than your speed and patience). If in doubt, take another one. Note that I count ``before lunch'' and ``after lunch'' as two separate bouts, so take a new calibration when you come back from breaks. 


\section{Tube Swabbing}
	The main purpose of this is to remove water in the tube, but you should also  swab if there seems to be any other crud affecting visibility on the inside of the tube. 
	\begin{enumerate}
		\item{Shine a flashlight down the tube. If there's no water and everything looks clean, skip swabbing and move on to image collection.}
		\item{If swabbing is needed, mount a roller onto the swab stick and extend the handle all the way. Make sure the swab is clean; grit and dirt will scratch the inside of the tube.}
		\item{Run the swab down the tube. Listen and feel for any splashing at the bottom.}
		\item{If the swab comes out dry, or perhaps very slightly damp from condensation, you're done. Skip to the imaging process.}
		\item{If the swab comes out wet enough to indicate standing water in the  tube, you're in for a few minutes of disgusting swabbing.}
		\item{Repeatedly run the swab down to the bottom of the tube, letting it soak up water and then removing it to wring it out. Switch off as needed between the long-pile roller, short-pile roller, and paper towels.}
		\item{When you're satisfied that the tube is thoroughly dry and clean, proceed to imaging. Pick up any paper towel waste, and do not put the swab down on anything gritty.}
\end{enumerate}
	
\section{Image Capture}
	Bear in mind that your main job during image collection is to make sure that the computer is always told accurately where it is. It has no way of double-checking you, and an image saved in the wrong place is worse than useless.
	
	There are innumerable quirks and bugs in the I-Cap system, so this section is intended as a reminder for the previously trained rather than as a step-by-step protocol.
	
\subsection{At Every Tube}
	\begin{enumerate}
		\item{Set the tube number correctly.}
		\item{Go to location 1. You should see at least part of the tube label. Use location 2 if the tube label is very clearly more visible there than at location 1.}
		\item{Convince yourself that the label is correct for the tube you think you're imaging, and that it matches the label in the reference image for that location (If you had to set up a new experiment, there will be no reference images, so be extra-cautious about the other checks).}
		\item{Make sure the image is in focus and well-lit.}
		\item{Save the image showing the label (pick either Location 1 or 2, even if the label is most visible halfway between them).}
		\item{Go to Location 5.}
		\item{Set up Autorun with a 5-location increment and not going past Location 110.}
		\item{Take a deep breath, get your hands ready, and hit control-U. Keep the camera in sync while it saves images for the remainder of the tube.}
	\end{enumerate}	

\subsection{Some General Reminders}

	When dismissing the Go window, avoid an unwanted image save by using Control-O, not the enter key.
	
	If anything seems wonky, save a note explaining it.
	
	Remember that notes are associated with images, not locations! If you make a note and then don't save an image at that location, the note isn't saved either.
	
	When using Go, beware the location-doesn't-update bug. It's okay to save an image at the new location when it's still displaying the old one, \textit{if} you're religiously sure you entered the new location correctly.
	
	If there are multiple images for a location and no notes to help me out, I will assume that the one captured later is correct and the one captured earlier is suspect. It's supposed to be quite difficult to accidentally capture a second image at an already-imaged location, but there seems to be a bug that lets it happen sometimes if you're not paying attention. Try to minimize this by working systematically and only setting the location back to an already-imaged one if it's necessary to correct a previous error.


\section{Equipment care}
	We're trying to haul a computer and a bunch of delicate optics around a dusty, pollen-infested field. Everything involved is hideously expensive. Scared? Me too. In general, treat everything as gently as you can and keep it as clean as it's practical to do.
	
	Avoid kinks and jerks on the camera cable. Disconnect it (turn off the case power first!) whenever you move the laptop.
	
	Don't put the camera down a tube that has water in it. It's less waterproof than they advertise. 
	
	If the camera head does get wet, turn it off immediately. Dry the outside with paper towels. Clean the lens area with alcohol and Q-tips. Clean the top ends of the light bulbs with alcohol, then wrap a bit of masking tape around them to get enough of a grip to pull them (gently) out of their sockets. Dry the insides of the sockets with the corners of a paper towel. Let everything air-dry thoroughly.
	
	If there's any sign at all that any water got inside of the camera head (say, fog on the inside of the lens glass), \textbf{do not turn the camera back on!} Bring it back to the lab and call Bartz for instructions.
	
	
If you have any trouble or something's not clear, give me a call (503 929-9421).  It's a good idea to coordinate scheduling with Sharon Gray (847-668-5279), and she can help with questions about the camera as well.		
\end{document}
	